\documentclass{article}	
\usepackage[margin=1in]{geometry}
\usepackage{fancyhdr}
\usepackage{duckuments}
\usepackage{enumitem}
\usepackage{graphicx}
\graphicspath{C:\Users\panda\.vscode\tex\MathfuerEcoII\MSE}
\usepackage{hyperref}
\usepackage{mathtools}
\usepackage{commath}
\pagestyle{fancy}
\fancyhead{} % clear all header fields
\fancyhead[LO]{\small{Math für Natur-und Wirtschaftswissenschaften}}
\fancyhead[RO]{\small{Basics of Multivariable Calculus}}
\fancyfoot{} % clear all footer fields
\usepackage{amsmath}
\begin{document}
\section*{Multivariable Calculus}
\subsection*{Basics}
analysis of multivariable functions: (generalizations to multi dimensions)
\begin{enumerate}
    \item continuity
    \item limits
    \item differentiability
    \item approximation
    \item optimization
    \item integration
\end{enumerate}
General concept: \newline
\begin{equation*} f: X \rightarrow Y \end{equation*}
\begin{equation*} \tag{multivariable} X \subseteq R^{n} \label{multivariable}\end{equation*} 
\begin{equation*} \tag{vector-valued} Y \subseteq R^{m} \label{vector-valued} \end{equation*}
\\[4pt]
let $f:X \rightarrow Y$ be a function with $X \subseteq R^{n}$ (domain) and $Y \subseteq R^{m}$ (codomain) for m,n $\in N$. The graph of f is the set: \\
\begin{equation*} \text{graph}(f) = \{\begin{pmatrix} x \\ f(x) \end{pmatrix} : x \in X \} \end{equation*} 
\subsection*{Basics of Topology}
In topology and related areas of mathematics, the neighbourhood system, complete system of neighbourhoods, or neighbourhood filter for a point $x$ in a \underline{topological space} is the collection of all \underline{neighbourhoods} of x \\
\textbf{basic definition:}
An open neighbourhood of a point $x$ in a topological space $X$ is any open subset $U$ of $X$ that contains $x$. A neighbourhood of $x$ in X is any subset $N \subseteq X$ that contains some open neighbourhood of $x$. In other words, $N$ is a neighbourhood of $x$ in $X$ if and only if there exists some open subset $U$ with $x \in U \subseteq N$.\textbf{A neighbourhood of $x$ is any set that contains $x$ in its topological interior}
\\ can refer to: \\ \url{https://www.math.tugraz.at/~ganster/lv_topologie_ss_2019/04_base_subbase_neighbourhood_base.pdf}
\subsection*{Definition:}
Let $x^\ast\in R^{n}, \delta \geq 0$. Then the \textbf{$\delta$-neighbourhood of x* is the set} 
\begin{equation*} N_{\delta}(x^\ast) = \{x\in R^{n}: \norm{x-x^\ast} < \delta \}\end{equation*} 
$N_\delta(x^\ast)$ is also called the \textbf{open ball of radius $\delta$ around $x^\ast$} and sometimes denoted by $B_\delta(x^\ast)$
\newpage
\subsection*{Definitions:} 
Let $X\subseteq R^{n}$ and $x^\ast \in X$
\begin{enumerate}
    \item $x\ast$ is called \textbf{interior point of X} if there is some $\delta > 0$  such that $N_\delta(x^\ast) \subseteq X$. The set of all interior points of X is called \textbf{interior of X} and denoted by \textbf{int$(X)$} or $\dot{X}$ 
    \item $x\ast$ is called \textbf{a boundary point of X} if each neighbourhood of $x^\ast$ contains both in $X$ and both points not in $X$, ie: \\
    $X \cap N_\delta(x^\ast) \neq \phi$ and $(R^n\backslash X) \cap N_\delta(x^\ast)\neq \phi$ for all $\delta > 0$ 
    \\ The set of all boundary points of X is called the \textbf{boundary/border of X} and denoted by \textbf{bd$(X)$} $(\text{or  }  \partial X)$. The set \textbf{cl$(X) = X \cup$ bd$(X)$} is called the \textbf{closure of $X$} (or $\bar{X}$)  
    \item The set X is called 
    \begin{enumerate}
        \item \textbf{open} if $X = \text{int}(X)$ 
        \item \textbf{closed} if $R\backslash X$ is open
        \item \textbf{bounded} if there is some $\delta > 0$ such that $X\subseteq N_\delta(0)$
        \item \textbf{compact} if it is closed and bounded
    \end{enumerate}

\end{enumerate}
\end{document} 
