\documentclass{article}	
\usepackage[margin=1in]{geometry}
\usepackage{fancyhdr}
\usepackage{duckuments}
\usepackage{enumitem}
\usepackage{graphicx}
\usepackage{hyperref}
\pagestyle{fancy}
\fancyhead{} % clear all header fields
\fancyhead[LO]{\small{Math für Natur-und Wirtschaftswissenschaften}}
\fancyhead[RO]{\small{Differential Equations}}
\fancyfoot{} % clear all footer fields
\usepackage{amsmath}
\begin{document}
\section*{Solving a Differential Equation}
\textbf{Case 1: Homogeneous DE}\\[2mm]
we have: 
\hspace{10mm}$i(t) = 0$ \hspace{5mm}and
\hspace{5mm}$k'(t) = \delta k(t)$ \\[2mm]
\text{Claim: solution is} $k(t)=\gamma\cdot e^{\delta t}$ \newline
\text{Set of solutions for the homogeneous DE:} $\{\gamma\cdot e^{\delta t} : \gamma \in R\}$ \\[3mm]
\textbf{Case 2: Inhomogeneous DE} \hspace{5mm} we have a general $i(t)$ 
\begin{enumerate}
    \item It is sufficient to determine (particular) solution of the Inhomogeneous DE \newline
    $K_n(t) \rightarrow$ solution for the homogeneous DE $\Rightarrow$ $K'_n (t) = \delta \cdot K_n (t)$ 
    \newline
    $K_p(t) \rightarrow$ solution for the Inhomogeneous DE $\Rightarrow$ $K'_p(t) = \delta \cdot K_p (t) + i(t)$
    \newline combining both Equations, we have: \newline
    \begin{equation*}
    \begin{split} 
    K'(t) & = K'_p(t)+K'_n(t) \\
          & = \delta \cdot K_p(t) + i(t) + \delta \cdot K_n(t) \\
          & = \delta \cdot \left(K_p(t)+K_n(t)\right) + i(t) \\
          & = \delta \cdot K(t) + i(t)
    \end{split} 
    \end{equation*}
    \item \text{Determine one solution for Inhomogeneous DE through variation of constant} \newline
    $k(t) = \gamma \cdot e^{\delta t}$ \newline
    $k(t) = c(t) \cdot e^{\delta t}$ \newline
    substituting this into the DE  \\[1mm]
    \begin{equation} k'(t) = c'(t) \cdot e^{\lambda t} + c(t) \cdot \delta e^{\delta t} \end{equation}
    \begin{equation} k'(t) = k(t) \cdot \lambda + i(t) \end{equation} 
    from (1) and (2) we have:
    \begin{equation*} c'(t) \cdot e^{-\lambda \cdot t} = i(t) \end{equation*} \newline
    Integrating both sides 
    \hspace*{30mm} \begin{equation*}c(t) = \int {i(t) \cdot e^{-\lambda t}dt} \end{equation*}
    \vspace{5mm}
\end{enumerate}
\textbf{Example:} Solving an Ordinary Differential Equation
\begin{equation*} f'(x) = 1 - x + f(x), \hspace{1mm} f(0) = 2 \end{equation*} 
\begin{enumerate} 
    \item \text{Homogeneous DE:} 
    \begin{equation*}f'(x) = f(x) \Rightarrow f(x) = \gamma \cdot e^{x} \end{equation*}
    check: \begin{equation*} f'(x) = f(x) = \gamma \cdot e^{x} - \gamma \cdot e^{x} = 0 \end{equation*}
    \item \text{Variation of constant:} 
    \begin{equation*} f(x) = c(x) \cdot e^{x} \end{equation*}
    \begin{equation*} f'(x) = c'(x) \cdot e^{x} + c(x) \cdot e^{x} \end{equation*} 
    \begin{equation*} f'(x) - f(x) = 1 - x \end{equation*}  
    \newpage
    Replacing $f'(x)$ and $f(x)$ with the function of $c(x)$ and $c'(x)$
    \begin{equation*} c'(x) \cdot e^{x} + c(x) \cdot e^{x} - c(x) \cdot e^{x} = 1 - x \end{equation*} 
    \begin{equation*}  c'(x) = e^{-x} \cdot (1-x) \end{equation*}
    \begin{equation*}  c(x) = \int e^{-x} \cdot (1-x) dx \end{equation*}
    \begin{equation*}  c(x) = \int e^{-x} dx - \int e^{-x} \cdot x dx \end{equation*}
    \begin{equation*}  c(x) = e^{-x} \cdot x \end{equation*}
\text{Particular solution:} \begin{equation*} f(x) = c(x) \cdot e^{x} \Rightarrow f(x) = e^{-x} \cdot x \cdot e^{x} \end{equation*}
\begin{equation*} f(x) = x \Rightarrow f'(x) = 1 \end{equation*}
    \item \text{Set of solutions:} \newline
          \begin{equation*} f(x) = x + \gamma \cdot e^{x} \end{equation*}
    \item Initial Value: \begin{equation*} f(0) = 2 \end{equation*} 
         \begin{equation*} f(0) = 0 + \gamma = 2 \Rightarrow \gamma = 2 \end{equation*}
         \begin{equation*} \rightarrow  f(x) = x + 2 \cdot e^{x} \end{equation*}
         \text{solves the initial value problem}
\end{enumerate} 
\textbf{Note: You can also solve the Differential Equation with the help of an integrating factor}\newline Refer to this wikipedia article \href{https://en.wikipedia.org/wiki/Integrating_factor}{integrating factor} and this \href{https://www.mathcentre.ac.uk/resources/uploaded/mathcentre-ode.pdf}{text}
\\[3mm]
\hspace*{70mm} \textbf{***} 
\subsection*{System of Differential Equations}
A homogeneous system of Linear Differential Equations with constant coefficients is a system of the form: \\[2mm]
\begin{equation*}\begin{pmatrix} y'_1(t) \\ \vdots \\ \vdots \\ y'_n(t) \end{pmatrix} = A \cdot \begin{pmatrix} y_1(t) \\ \vdots \\ \vdots \\ y_n(t) \end{pmatrix}, \hspace{5mm} \begin{pmatrix} y_1(0) \\ \vdots \\ \vdots \\ y_n(0) \end{pmatrix} = \begin{pmatrix} \gamma_1 \\ \vdots \\ \vdots \\ \gamma_2 \end{pmatrix}\end{equation*} \\[2mm]
\hspace*{45mm} where A $\in R$ and $\begin{pmatrix} \gamma_1 \\ \vdots \\ \vdots \\ \gamma_2 \end{pmatrix} \in R^{n} $ \\[2mm]
The key idea is that if A is a diagonal matrix, then each row is a Differential Equation that is independent of the others, solve each row separately \textbf{decoupled system}
\begin{equation*} \begin{pmatrix} y'_1(t) \\ y'_2(t) \end{pmatrix} = \begin{pmatrix} \lambda_1 & 0 \\ 0 & \lambda_2 \end{pmatrix} \begin{pmatrix} y_1(t) \\ y_2(t) \end{pmatrix}  \end{equation*}
We need a basis and a corresponding basis transformation S $\in R^{n \times n}$ such that $S^{-1}AS = D$ is diagonal \newline
Then we substitute, y = SZ $\Rightarrow Z = S^{-1}y$ \newline
The DE system, \begin{equation*} y' = A \cdot y \end{equation*} 
becomes \begin{equation*} S \cdot z = A \cdot S \cdot Z \end{equation*} 
\begin{equation*} Z' = S^{-1} \cdot A \cdot S \cdot Z = D \cdot Z \end{equation*}
Solve the new decoupled system and transform solution back through $y = SZ$ \newline
\textbf{Example:} 
\begin{equation*} \gamma'(t) = \frac{5}{2} \cdot \gamma(t) - \phi(t) \end{equation*}
\begin{equation*} \phi'(t) = \frac{-1}{4} \cdot \gamma(t) + \frac{5}{2} \phi(t) \end{equation*}
matrix form \begin{equation*} \begin{pmatrix} \gamma' \\[6pt] \phi' \end{pmatrix} = \begin{pmatrix} \dfrac{5}{2} & -1 \\[9pt] \dfrac{-1}{4} & \dfrac{5}{2} \end{pmatrix} \begin{pmatrix} \gamma \\[6pt] \phi \end{pmatrix}\end{equation*}
now, determine the eigenvalues and eigenvectors for the matrix\newline
\begin{equation*} p_A(\lambda) = \det(A-\lambda I_2) = \det\begin{pmatrix} \dfrac{5}{2} - \lambda & -1 \\[9pt] \dfrac{-1}{4} & \dfrac{5}{2} - \lambda \end{pmatrix} = {\left(\frac{5}{2}-\lambda\right)}^{2} - \frac{1}{4} = \frac{25}{4} + \lambda^{2} - 5\lambda - \frac{1}{4} = 6 + \lambda^{2} -5\lambda = (\lambda-3)(\lambda-2) \end{equation*}
The roots of the equation are $\lambda_1 = 3$ and $\lambda_2 = 2$ \newline
For the respective eigenspaces, we get \newline
\begin{equation*} eig_A(3) = \ker(A-3I_2) = \ker\begin{pmatrix} \dfrac{-1}{2} & -1 \\[9pt] \dfrac{-1}{4} & \dfrac{-1}{2}\end{pmatrix} = \ker\begin{pmatrix} \dfrac{1}{2} & -1 \\[9pt] 0 & 0\end{pmatrix} = span\begin{Bmatrix}\begin{pmatrix} -2 \\1 \end{pmatrix}\end{Bmatrix} \end{equation*}
\begin{equation*} eig_A(2) = \ker(A-2I_2) = \ker\begin{pmatrix} \dfrac{1}{2} & -1 \\[9pt] \dfrac{-1}{4} & \dfrac{1}{2}\end{pmatrix} = \ker\begin{pmatrix} \dfrac{1}{2} & -1 \\[9pt] 0 & 0\end{pmatrix} = span\begin{Bmatrix}\begin{pmatrix} 2 \\1 \end{pmatrix}\end{Bmatrix} \end{equation*}
\begin{equation*} S = \begin{pmatrix} -2 & 2 \\ 1 & 1 \end{pmatrix}\end{equation*}
\begin{equation*} S^{-1} = \frac{-1}{4}\begin{pmatrix} 1 & -2 \\ -1 & -2 \end{pmatrix}\end{equation*}
\begin{equation*} S^{-1}AS = \begin{pmatrix} 3 & 0 \\ 0 & 2 \end{pmatrix} = D \end{equation*}
Solve the system: \begin{equation*} Z' = DZ = \begin{pmatrix} 3 & 0 \\ 0 & 2 \end{pmatrix} \end{equation*}
\begin{equation*} Z_1(t) = \gamma_1 e^{3t} \end{equation*}
\begin{equation*} Z_2(t) = \gamma_2 e^{2t} \end{equation*}
\\[2mm]
\begin{equation*} \begin{pmatrix} \gamma \\[6pt] \phi  \end{pmatrix} = \begin{pmatrix} -2 & 2 \\[6pt] 1 & 1 \end{pmatrix} \begin{pmatrix} \gamma_1 e^{3t} \\[6pt] \gamma_2 e^{2t} \end{pmatrix} \end{equation*}
\begin{equation*} \gamma(t) = -2 \gamma_1 e^{3t} + 2 \gamma_2 e^{2t} \end{equation*}
\begin{equation*} \phi(t) = \gamma_1 e^{3t} \gamma_2 e^{3t} \end{equation*}
we know the initial values: $\gamma(0) = 60$ and $\phi(0) = 60$ \newline
\begin{equation*} -2\gamma_1 + 2\gamma_2 = 60\end{equation*}
\begin{equation*} \gamma_1 + \gamma_2 = 60 \end{equation*} 
solving, we get $\gamma_1 = 15$ and $\gamma_2 = 45$
\begin{equation*} \gamma(t) = -30 e^{3t} + 90 e^{2t} \end{equation*}
\begin{equation*} \phi(t) = 15 e^{3t} + 45 e^{2t} \end{equation*}
\vfill
Written by Prajeet Pushkar (TUM Bachelor, SS24), \href{https://github.com/PenguinPuff}{my github!} 
\end{document}
